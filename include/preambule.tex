
\usepackage[hmargin=2.5cm, vmargin=2.5cm]{geometry}
\usepackage{fancyhdr}

\usepackage{pdflscape}

\usepackage{mathptmx}
\usepackage{color}
\usepackage{setspace}
\usepackage{multicol}

\usepackage{graphicx} % Pour insérer des images
\usepackage{wrapfig}  % Pour insérer des images dans des paragraphes
\usepackage{float}    % Pour utiliser floatplacement

% Packages pour les tableaux
\usepackage{colortbl} % Pour colorier les tableaux
%\usepackage{multirow} % Pour fusionner des lignes
%\usepackage{slashbox} % Pour faire des bordures diagonales
%\usepackage{array}    % Pour utiliser !{séparateur} et @{séparateur}
%\usepackage{tabularx} % Pour faire des tableaux de la largeur de la page

% Packages pour le code
%\usepackage{verbatim}     % Pour écrire des blocs de code
%\usepackage{moreverb}     % Pour indenter les blocs de code
\usepackage{listingsutf8} % Pour mettre en forme du code

% Autres packages
\usepackage{url}      % Pour mettre des liens hypertextes
\usepackage{eurosym}  % Pour utiliser le symbole euro avec \euro{}
\usepackage{layout}   % Pour afficher les marges du document avec \layout
\usepackage{lastpage} % Pour avoir le nombre total de pages
\usepackage{hyperref} % Permet de rendre le PDF navigable

\usepackage{etoolbox} % ifdefempty

\usepackage{pgffor}

% Définition de la coloration syntaxique
%\lstset{language=sh}
%\lstset{tabsize=4}
%\lstset{frame=single}
%\lstset{breaklines=true}
%\lstset{showstringspaces=false}
%\lstset{basicstyle=\footnotesize}
%\lstset{escapeinside={>>*}{*<<}}
%\lstset{inputencoding=utf8/latin1}
%\lstset{deletekeywords={if, for, read, unset, history}}

% Définition des en-têtes et pieds de page
\setlength{\headheight}{15pt}
\fancypagestyle{toc-style}
{
	\renewcommand{\headrulewidth}{1pt}
	\renewcommand{\footrulewidth}{1pt}
	\lhead{\textbf{\csname @title\endcsname}}
	\chead{}
	\rhead{}
	\lfoot{\textit{\csname @author\endcsname}}
	\cfoot{}
	\rfoot{\textbf{Table des matières}}
}

\fancypagestyle{page-style}
{
	\renewcommand{\headrulewidth}{1pt}
	\renewcommand{\footrulewidth}{1pt}
	\lhead{\textbf{\csname @title\endcsname}}
	%\lhead{\textbf{\hyperref[Sommaire]{\csname @title\endcsname}}} % Avec hyperref
	\chead{}
	\rhead{}
	\lfoot{\textit{\csname @author\endcsname}}
	\cfoot{}
	\rfoot{\textbf{Page \thepage{} sur \pageref{LastPage}}}
	%\rfoot{\textbf{Page \thepage{} sur \pageref*{LastPage}}} % Avec hyperref
}

% Définition des couleurs
\definecolor{grisClair}{rgb}{0.75, 0.75, 0.75}
%\setlength\parindent{0pt}


\usepackage{listings} 
\usepackage{inconsolata} 
\usepackage{blindtext,expdlist} 

\lstdefinelanguage{CISCO} 
{ 
  sensitive=true, 
  morekeywords=[1]{},
  tabsize=4,
  frame=single,
  breaklines=true,
  showstringspaces=false,
  basicstyle=\ttfamily\scriptsize,
  escapeinside={>>*}{*<<},
  inputencoding=utf8/latin1,
} 
